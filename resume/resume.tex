\documentclass[11pt,a4paper,roman]{moderncv} % Font sizes: 10, 11, or 12; paper sizes: a4paper, letterpaper, a5paper, legalpaper, executivepaper or landscape; font families: sans or roman

\moderncvstyle{casual} % CV theme - options include: 'casual' (default), 'classic', 'oldstyle' and 'banking'
\moderncvcolor{blue} % CV color - options include: 'blue' (default), 'orange', 'green', 'red', 'purple', 'grey' and 'black'

\usepackage{color}
\usepackage{lipsum} % Used for inserting dummy 'Lorem ipsum' text into the template

\usepackage[scale=0.75]{geometry} % Reduce document margins
%\setlength{\hintscolumnwidth}{3cm} % Uncomment to change the width of the dates column
%\setlength{\makecvtitlenamewidth}{10cm} % For the 'classic' style, uncomment to adjust the width of the space allocated to your name

%----------------------------------------------------------------------------------------
%	NAME AND CONTACT INFORMATION SECTION
%----------------------------------------------------------------------------------------

\firstname{Alankar} % Your first name
\familyname{Kotwal} % Your last name

% All information in this block is optional, comment out any lines you don't need
\title{Detailed Resume}
\email{alankar.kotwal@iitb.ac.in}
\homepage{alankarkotwal.github.io}{alankarkotwal.github.io}
\mobile{+91-9969678123}
\extrainfo{\url{www.github.com/alankarkotwal}}
\photo[70pt][0.4pt]{pic.jpg}

\begin{document}
\makecvtitle

\section{Education}

\cventry{2012--Present}{Dual Degree, B. Tech and M.Tech in Electrical Engineering}{\newline Indian Institute of Technology}{Bombay}{\textit{CPI -- 8.92/10}}{Specialization: Communication and Signal Processing, Minor Degree: Computer Sciences and Engineering}
\cventry{2010--2012}{Intermediate Examination}{\newline Ratanbai Walbai Junior College of Science}{Mumbai}{\textit{Percentage -- 93.83}}{}
\cventry{2001--2010}{Matriculation}{\newline SVPT's Saraswati Vidyalaya}{Thane}{\textit{Percentage -- 95.27}}{}

\section{Achievements}
\cventry{2012}{\textbf{Gold Medal, International Olympiad on Astronomy and Astrophysics}}{\newline Brazil} {International Rank 4, Special Prize for Best Data Analysis}{}{}
\cventry{2011}{\textbf{Bronze Medal, International Earth Sciences Olympiad}}{\newline Italy} {Special Prize for Best Performance in Hydrosphere section}{}{}
\cventry{2012}{All India Rank 105}{IIT-JEE}{\newline among around 5,90,000 participants for entrance to the IITs}{}{}
\cventry{2009--2012}{Olympiad Orientation-cum-Selection Camps}{}{\newline Selected for the following camps, among the top 30 students in India (Astronomy: 2012 \& 2010, Earth Sciences: 2011, Junior Sciences: 2010 \& 2009)}{}{}
\cventry{2010}{Kishore Vaigyanik Protsahan Yojana Scholarship}{}{\newline Awarded by the Government of India to students interested in research}{}{}
\cventry{2008}{National Talent Search Examination Scholarship}{}{\newline Awarded by the Government of India to students interested in research}{}{}
\cventry{2011--2012}{Infosys Award for Olympiad Medallists}{}{}{}{}
\cventry{2013}{Inter-IIT Messier Marathon}{}{\newline Secured IIT Bombay the second position by putting on board 72 messier objects including the entire Virgo cluster of galaxies}{}{}
\cventry{2013}{Other competitions}{}{\newline Won the Innovation Cell recruitment contest for freshmen and the Astronomy Quiz conducted by the Astronomy Club, IITB in 2012 and BITS Goa in 2013}{}{}

\newpage

\section{Experience: Electrical Engineering and Computer Sciences}

\cventry{2014--Present}{Fourier Ptychographic Microscopy}{\newline The LV Prasad Eye Institute}{}{}{
\begin{itemize}
\item Studied and implemented the technique named Fourier Ptychographic Microscopy for wide-field, high-resolution static imaging
\item Analyzed the physics of the system for the case of transmissive imaging in detail using Fourier optics.
\end{itemize}
}

\cventry{2014}{Google Summer of Code}{}{\newline See the Astronomy and Astrophysics section below}{}{}

\cventry{2013--Present}{Computer Vision, The IITB Mars Rover Team}{\newline A Student Initiative at IITB}{}{}{
\begin{itemize}
	\item{Work in 2014:}
	\begin{itemize}
		\item{Exploring illumination-corrected stereo vision and shape from motion for autonomous navigation}
		\item{Design and testing of a new algorithm for navigation and obstacle avoidance}
		\item{Implementation of the rover software stack on ROS}
	\end{itemize}
	\item{Work in 2013:}
	\begin{itemize}
		\item{Programming manual controls and safety on-board}
		\item{Hardware interfacing for peripherals on-board and debugging}
	\end{itemize}
\end{itemize}
}

\cventry{2014}{The Arkaroola Mars Robot Challenge}{\newline A joint venture of the Mars Society Australia and Saber Astronautics}{}{}{
\begin{itemize}	
\item{Tested the Mars Rover prototype developed by the IITB Rover Team in the harsh conditions of the Australian outback}
\item{Participated in a series of exercises in Mars operations research conducted by Saber Astronautics which included simulated extra-vehicular activities in simulated space-suits}
\item{Explored Arkaroola geology and studied its similarities to Martian geology}
\end{itemize}
\textbf{Featured in Clarke et al., "Field Robotics, Astrobiology and Mars Analogue Research on the Arkaroola Mars Robot Challenge Expedition", Australian Space Research Conference 2015}
}

\cventry{2014}{Gravitational Lens Identification Using Image Processing Techniques}{\newline A PCA-based Method for Identifying Lenses in Databases}{\newline Prof. A. Rajwade and S. Awate, Department of Computer Sciences}{\newline Indian Institute of Technology Bombay}{
\begin{itemize}
\item{Improvised on source-subtraction algorithms for lens subtraction}
\item{Implemented the algorithm in Matlab and got a good identification rate lenses}
\end{itemize}
}

\cventry{2014}{Microprocessor Design}{\newline Design, Implementation and Validation of Three Processors in Verilog}{\newline Prof. V. Singh, Department of Electrical Engineering}{\newline Indian Institute of Technology Bombay}{
\begin{itemize}
\item{Designed and simulated a pipelined processor with the Little Computer Architecture}
\item{Designed, implemented and tested a multi-cycle RISC processor using the LC-3b ISA}
\item{Designed a CISC processor with reduced 8085 architecture}
\end{itemize}
}

\cventry{2014}{Temperature Controller on a CPLD}{\newline A Peltier-Plate Based Fast-Response P-Controller for Temperature Control}{\newline Prof. J. Mukherjee, Department of Electrical Engineering}{\newline Indian Institute of Technology Bombay}{}


\section{Experience: Astronomy and Astrophysics}
\cventry{2014}{Google Summer of Code}{\newline A New Pixel-Level Method for Determination of Photometric Redshifts}{\newline Prof. R. J. Brunner and M. C. Kind, Laboratory for Cosmological Data Mining}{University of Illinois at Urbana-Champaign}{
\begin{itemize}
\item{Developed the software package image-photo-z implementing this new method}
\item{Worked with SDSS photometry data and extracted pixel-level information for training machine learning algorithms: k-nearest neighbour algorithm and trees for photo-z}
\item{Worked on parallel programming and performance enhancement for this method}
\item{Validated the approach and got consistent predictions for redshifts in the testing set}
\end{itemize}
}

\cventry{2013}{National Initiative for Undergraduate Studies -- Astronomy}{\newline An X-Ray Study of Black Hole Candidate X Norma X-1}{\newline Prof. Manojendu Choudhury}{Center for Basic Sciences, University of Mumbai}{
\begin{itemize}
\item{Analysed statistically timing information from RXTE to detect quasi-periodic oscillations and find their possible relation to accretion disk thickening and synchrotron jets}
\item{Fitted the spectra obtained with a thermal and non-thermal power-law distribution to obtain essential system parameters and observed unusual oscillations in the inner radius}
\item{Working on finding a possible cause for these oscillations}
\end{itemize}
}

\cventry{2012}{National Initiative for Undergraduate Studies -- Astronomy}{\newline Estimation of Photometric Redshifts Using Machine Learning Techniques}{\newline Prof. Ninan Sajeeth Philip}{Inter University Center for Astronomy and Astrophysics, Pune}{
\begin{itemize}
\item{Estimated redshift data from colour index information obtained from SDSS data artificial neural networks}
\item{Worked on generation of training data from available data by redshifting spectra}
\end{itemize}
}

\cventry{2013-2014}{Resource Person}{\newline Indian National Astronomy Olympiad Programme}{}{\newline Homi Bhabha Center for Science Education}{
\begin{itemize}
\item{Selected twice as a Student Facilitator and a Resource Person for the Indian Astronomy Olympiad OCSC (Orientation-Cum-Selection Camp) for mentoring camp students, handling academic and organizational arrangements and aiding in evaluations}
\item{Involved in the selection and rigorous training of the 3 member Indian National team which won 3 Gold Medals at the International Astronomy Olympiad 2013 held in Lithuania}
\item{Involved in generating problems for the Indian National Astronomy Olympiad which is conducted as a part of selection of students for the camp}
\end{itemize}
}

\cventry{2014}{Gravitational Lens Identification}{}{\newline See the Computer Sciences section above}{}{}

\newpage

\section{Research Interests}
\cventry{}{Electrical Engineering and Computer Sciences}{}{}{}{
\begin{itemize}
\item{Using computational Fourier Optics for imaging resolution improvement}
\item{3D shape reconstruction using computer vision techniques}
\item{Robot navigation using stereo vision and structure from motion}
\item{Efficient algorithms for robot navigation using geometry of visual field}
\item{Processor architecture}
\item{Hardware description and simulation}
\end{itemize}
}

\cventry{}{Astronomy and Astrophysics}{}{}{}{
\begin{itemize}
\item{Cosmology and the large-scale structure of the universe}
\item{Stellar populations, structure and evolution}
\item{Applications of computer vision to astronomy}
\item{Data mining and its applications for handling astronomical data}
\end{itemize}
}

\cventry{}{Things I'd like to do}{}{}{}{
\begin{itemize}
\item{Logic minimization}
\item{Operations research in relation to Mars missions}
\end{itemize}
}

\section{Relevant Skills}
\cventry{}{Languages}{}{}{}{C/C++, Python, Shell Scripting, Java, Matlab, SQL, HTML/CSS, PHP, \LaTeX}
\cventry{}{Science Software}{}{}{}{Python packages: NumPy, SciPy and Matplotlib, GNUPlot, Scikit-learn, Astropy, SExtractor, SDSS tools}
\cventry{}{Special Software}{}{}{}{ROS/Gazebo, OpenCV, The Point Cloud Library, SPICE Circuit Simulation, EAGLE PCB Design, SolidWorks CAD, AutoCAD, LabView, Django}
\cventry{}{Hardware}{}{}{}{Microprocessor Architectures: 8051, 8085, AVR and PIC, CPLDs and FPGAs, Embedded Platforms: Arduino, RaspberryPi, Beaglebone, and so on, standard digital logic families}

\section{Relevant Courses Undertaken}

\cventry{}{Computer Sciences}{}{}{}{Computer Vision, Algorithms for Medical Image Processing, Machine Learning, Convex Optimisation, Digital Image Processing, Design and Analysis of Algorithms, Data Structures and Algorithms, Discrete Mathematics}

\cventry{}{Electrical Engineering}{}{}{}{Digital Signal Processing, Controls, Probability and Random Processes, Digital Communication, Communication Systems, Microprocessors, Signals and Systems, Digital and Analog Systems, Electrical Machines and Power Electronics, Power Systems, Electronic Devices and Circuits, Network Theory}

\cventry{}{Physics and Mathematics}{}{}{}{The General Theory of Relativity, Quantum Mechanics I*, Statistical Physics*, Electromagnetic Waves, Electricity and Magnetism, Classical Mechanics, Differential Equations, Linear Algebra, Complex Analysis, Calculus}

\end{document}